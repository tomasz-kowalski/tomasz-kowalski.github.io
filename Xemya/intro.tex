\documentclass{amsart}
\usepackage{tikz}

\title{Strategic games on Xemya}
\author{Tomasz Kowalski}


\begin{document}
\maketitle

\section{The story so far}

Consider planet Xemya. On Xemya, just like on Earth,
there are countries, and believe it or not, it is not unheard of for 
Xemyans to wage war. There are two neighbouring
countries on Xemya, called Apollonia and Tysq. Tysq is a
  major power, Apollonia is not.
  Other Xemyan powers include Albania\footnote{By a strange coincidence there is
    a country called Albania on 
  Xemya, with a reputation of not always being perfectly honest in her
  dealings with the neighbours. This is how some self-critical Albanians put it.
The neighbours, especially from Tysq, put it slightly differently: they went so
far as to coin a phrase ``perfidious Albania''.}, Coqauvin, Ruritania, Sipango,
and Ammer-Ku. 

At some point in troubled Xemya's history, Tysq goes nasty and demands that
Apollonia hand over to them a spaceport, over which it---Apollonia that is---has
some sort of dubious half-authority, by way of exercising a mandate of an international
peace-keeping organisation (whose peace-keeping successes are comparable with
those of UN on Earth). Tysq's diplomats left
Apollonians in no doubt that should they fail to meet their demand, bad things
of the military sort are sure to happen to Apollonia. 

This worries Apollonian strategists quite a lot. They are well aware
of the fact that Apollonia is no match for Tysq, which is quickly growing
stronger and more aggresive. Not 
a year has passed since Tysq gave a similar ultimatum to Tscheslavia.
Under diplomatic pressure, mainly from Albania and Coqauvin,
Tscheslavia duly complied, yet it was invaded and
occupied by Tysq. 
  
Apollonia hastily arranges some alliances, important ones with Albania and
Coqauvin. Indeed, Albania and Coqauvin propose an alliance themselves,
stating their unvawering support for Apollonia and encouraging her to stand firm. 
This gives hope to Apollonian populace, but the strategists are not all that
optimistic. It is their job to consider worst-case scenarios, and such scenarios
are looking grim. To begin with, a great war seems inevitable.
There is just too much tension around:
\begin{itemize}  
\item Ruritania claims she has a mission to liberate every oppressed Xemyan.
      To accomplish this worthy goal, she has to conquer the whole planet.
\item Tysq and Sipango are sworn enemies of Ruritania. They have their
      own way of making xemyankind evolve into a better species.
      By force, if neccessary.
\item Albania, Coqauvin and Ammer-Ku are happy to let everyone else
    be, as long as they engage in free trade. On their terms. 
\item Albania and Coqauvin have recently defeated Tysq. She
    resents it, and is getting ready for revenge.
\item Resource-hungry Sipango feels suffocated by a blockade imposed by
      Ammer-Ku and Albania, and needs to do something about it.
\item Finally, both Tysq and Ruritania seem to be convinced, each for
      different reasons, that Xemya without the nuisance of Apollonia would
      be a better Xemya.   
\end{itemize}

Against this pleasant background, Apollonia's strategists set out to
work. The question is: accept or reject?

Acceptance means surrender, but at this cost Apollonia could
perhaps avoid an immediate disastrous war. \emph{Perhaps}, because Tysq's
real intentions may not be as minimalistic as the ultimatum suggests.

Rejection means war, but an alliance of Apollonia, Albania and Coqauvin
theoretically has enough power to defeat Tysq. \emph{Theoretically},
because military alliances on Xemya have an unpleasant tendency to be unreliable
and unstable.
  
What to do? Where conventional wisdom fails, unconventional wisdom
is available, in the form of an oracle. To the oracle the strategists
go.

\vskip1cm
\hrule
\bigskip\noindent
\textsc{Now go to the oracle yourself. Hit the `back' button and scroll down.}
\bigskip
\hrule

\end{document}

%%% Local Variables:
%%% mode: latex
%%% TeX-master: t
%%% End:
